\chapter{Blockade Shift Calculations}\label{app_udd}
The Rydberg Blockade shift arises from the large dipole-dipole interactions between Rydberg-Rydberg moelcular states.
In general the magnitude of the van der Waals effect will scale as $R^{-6}$, where $R$ is the interactomic distance.
However, in the case of a second (or more) nearly-degenerate, dipole-allowed Rydberg-Rydberg molecular state(s), $2E(|nj\rangle) \sim E(|n_{1}j_{1}\rangle) + E(|n_{2}j_{2}\rangle)$, F\"{o}rster coupling between the molecular states results in an enhanced energy shift in the doubly excited $|nj\rangle + |nj\rangle$ state that scales as $R^{-3}$ at close distances.

The dipole-dipole operator $V_{dd}$, where the internuclear separation axis is along $z$,is:
\begin{equation}
  \label{eq_vdd}
  V_{dd}(R) = -\frac{\sqrt{6}e^2}{R^3} \sum\limits_p C^{20}_{1p1\bar{p}}a_p b_{\bar{p}},
\end{equation}
where $a_p(b_p)$ is the position of atom $a(b)$'s electron \cite{WS2007}.
Rienhard has the following formula\cite{Rienhard2007}:

\begin{equation}
  \label{eq_vdd_reinhard}
%  \begin{align}
  V_{dd}(R, \theta ) =
 %	-\frac{e^2}{R^3}\Bigl[ \\
%		&\qquad a_1b_{\bar{1}} + a_{\bar{1}}b_1 + a_0b_0 (1-3 \cos^2 \theta ) \\
%		&\qquad - \frac{3}{2}\sin^2\theta(a_1b_1 + a_1b_{\bar{1}} + a_{\bar{1}}b_1 + a_{\bar{1}}b_{\bar{1}} ) \\
%		&\qquad - \frac{3}{\sqrt{2}}\sin\theta\cos\theta(a_1b_0 + a_{\bar{1}}b_0 + a_0b_1 + a_0b_{\bar{1}} ) \\
	\Bigr],
%  \end{align}
\end{equation}
which does not reduce to Equation \ref{eq_vdd} when $\theta=0$ (there is a minus sign discrepancy in the $a_0b_0$ term, which since we are restricted to $\Delta l=\pm 1$ by the dipole parity of operator shouldn't matter here but I still need to figure it out)  The problem is that Georg uses a non-standard special tensor definition.
Since $\Delta \ell \neq 0$ and we are dealing with the stretched state any term w, $a_pb_0$

If we assume $j - j_1 = 1$, then the dipole-dipole molecular coupling matrix element is:
\begin{equation}
  \label{eq_vddcoupling}
  \begin{split}
    \langle\psi'|V_{dd}|\psi\rangle 
      &= \frac{1}{\sqrt{2}}\bigg( {_a}\langle n_1 j_1 m_{j1}| {_b}\langle n_2 j_2 m_{j2}| + {_a}\langle n_2 j_2 m_{j2}| {_b}\langle n_1 j_1 m_{j1}| \bigg) V_{dd}|\psi\rangle \\
      &= -2\sqrt{3}\times\frac{e^2}{R^3}\sum_pC_{1p1\bar{p}}^{20}\bigg({_a}\langle n_1 j_1 m_{j1}| a_p|njm_j\rangle_a\bigg)\bigg({_b}\langle n_2 j_2 m_{j2}| b_{\bar{p}}|njm_j\rangle_b\bigg)\\
      &\equiv \frac{C_3}{R^3},
  \end{split}
\end{equation}
where $C_3$ is the electric-dipole coupling coefficient between the two molecular Rydberg states.
Writing Equation~\ref{eq_vddcoupling} in the $\cal LS$-basis and applying the electric-dipole selection rules ($|\ell_{(1,2)}-\ell|=1$, $m_{l1}=m_l+p$, and $m_{s1}=m_s$) results in:
\begin{equation}
  \label{eq_vddls}
  \begin{split}
    V_{dd}(R) &= -2\sqrt{3}\times\frac{e^2}{R^3}\sum_{p,m_{\ell}} \Bigl[ C_{1p1\bar{p}}^{20} \left(C_{\ell m_{\ell}\frac{1}{2}(m-m_{\ell})}^{jm}\right)^2 \\
                &\qquad \times C_{\ell_{1}(m_{\ell}+p)\frac{1}{2}(m_1-m_{\ell}-p)}^{j_1m_1}\langle n_1l_1(m_{\ell}+p)|r_p|n\ell m_{\ell}\rangle\\
                &\qquad \times C_{\ell_{2}(m_{\ell}-p)\frac{1}{2}(m_2-m_{\ell}+p)}^{j_2m_2}\langle n_2l_2(m_{\ell}-p)|r_{\bar{p}}|n\ell m_{\ell} \rangle \Bigr]
  \end{split}
\end{equation}
Under the special condition where $j=l+\frac{1}{2}$ this simplifies to:
\begin{equation}
  \label{eq_vddls}
    V_{dd}(R) = -\sqrt{2}\frac{e^2}{R^3}
               \langle n_1 \ell_1|| r || n \ell \rangle \langle n_2 \ell_2 ||r || n \ell \rangle
\end{equation}

Therefore the F\"{o}rster coupling between the two molecular states is $V = C_3/R^{3}$ and the Hamiltonian is:
\begin{equation}
  \label{eq_hdd}
  H_{dd} =  
    \left( 
      \begin{array}{cccc}
        0      & V_1      & V_2      & \cdots \\
        V_1    & \delta_1 & 0        & \\
        V_2    & 0        & \delta_2 & \\
        \vdots &          &          & \ddots
      \end{array} 
    \right),
\end{equation}
where $\delta_i$ is the F\"{o}rster defect (molecular state energy difference) of the $i^{th}$ molecular state.
+
In the limit of a single molecular state with a small F\"{o}rster defect the dipole-dipole shift of the doubl excited state becomes:
\begin{equation}
  \label{eq_vddsimple}
  U_{dd}(R) = \frac{\delta}{2}\left(1 - \sqrt{1 + \left(\frac{2C_3}{\delta R^{3}}\right)^2}\right).
\end{equation}
In the limit $\delta R^{-3}\gg C_3$ there is weak coupling and we obtain the expected van der Waals behavior $U_{dd}(R) \approx C_3^2/\delta R^{6} \equiv C_6/R^{6}$.
However, for $\delta R^{-3}\ll C_3$ the interaction is now $U_{dd}(R) \approx \frac{\delta}{2} - \frac{C_3}{R^{3}}$.
The effect of a F\"{o}rster resonance, $\delta \sim 0$, is to increase the critical distance $R_c = C_3^{1/3}$ by a factor of $\delta^{-1/3}$ where the characteristic of the potential changes from $R^{-3}$ to $R^{-6}$.
Careful selection of Rydberg levels with F\"{o}rster resonances in mind can decrease the principle quantum number $n$ required to realize a sufficient blockade strength.

\section{$97D_{5/2}, m_j = 5/2$}

\subsection{First Term}
The nearly degenerate molecular states which contribute the most to the blockade shift for the first term in Equation \ref{eq_vdd_reinhard} are:
\begin{equation}
  \label{eq_rydmol1}
  97D_{5/2} + 97D_{5/2} \leftrightarrow 99P_{3/2} + 95F_{7/2}
\end{equation}
and
\begin{equation}
  \label{eq_rydmol1}
  97D_{5/2} + 97D_{5/2} \leftrightarrow 98P_{3/2} + 96F_{7/2}.
\end{equation}
Using Mark's Rydebrg energy level calculations \cite{}, the F\"{o}rster defects are $\delta_{(1,2)}=(150,238)$ MHz respectively.

Since we are only exciting the stretched $|nD_{5/2},m_j=5/2\rangle$ state the only electric-dipole allowed fine-structure coupling is to the $|n_1P_{3/2},m_j=3/2\rangle + |n_1F_{7/2},m_j=7/2\rangle$ molecular state, which are both also stretched states.
The radial matrix elements for these states are listed below in units of a${_0}$ ($\langle n'\ell '||r||n\ell\rangle \equiv \sqrt{2\ell +1}C^{\ell '0}_{\ell 010}R^{n'\ell '}_{n\ell}$) \cite{WS2007}:
\begin{subequations}
  \label{eq_dradmatelem}
  \begin{align}
  \langle 99P_{3/2}||r|97D_{5/2} \rangle
	=\sqrt{2(1)+1}C^{20}_{1010}R^{97D}_{99P}\\
	=0.73\sqrt{2}\times 99^2 a0,\\
    \langle 95F_{7/2}||r||97D_{5/2} \rangle 
	=\sqrt{2(2)+1}C^{30}_{2010}R^{95F}_{97D}\\
	=0.8\sqrt{3}\times 97^2,\\
    \langle 98P_{3/2}||r||97D_{5/2} \rangle 
	=\sqrt{2(1)+1}C^{20}_{1010}R^{97D}_{98P}\\
	= 1.3\sqrt{2}\times 98^2,\\
    \langle 96F_{7/2}||r||97D_{5/2}\rangle 
	=\sqrt{2(2)+1}C^{30}_{2010}R^{96F}_{97D}\\
	= 1.3\sqrt{3}\times 97^2.
  \end{align}
\end{subequations}
Therefore the dipole-dipole coupling term, $C_3$, term for each molecular state is:
\begin{equation}
  \label{eq_dc3s}
  \begin{split}
    C_3^{(1)} 
      &= -\sqrt{2}e^2 \langle 99P_{3/2}, m_{\ell}=1|r_{-1}|97D_{5/2}, m_{\ell}=2\rangle \langle 95F_{7/2}, m_{\ell}=3|r_1|97D_{5/2}, m_{\ell}=2\rangle\\
      &= -\sqrt{2}e^2 \times (0.73\sqrt{2}\times 99^2 \ a_0) \times (0.8\sqrt{3}\times 97^2 \ a_0) \\
      &=-175600 \text{ MHz} \cdot \mu \text{m}^3,\\
    C_3^{(2)} 
      &= -\sqrt{2}e^2 \langle 98P_{3/2},m_{\ell}=1|r_{-1}|97D_{5/2},m_{\ell}=2\rangle \langle 96F_{7/2},m_{\ell}=3|r_1|97D_{5/2},m_{\ell}=2\rangle\\ 
      &= -\sqrt{2}e^2 \times (1.3\sqrt{2}\times 98^2 \ a_0) \times (1.3\sqrt{3}\times 97^2 \ a_0) \\
      &=-498100 \text{ MHz} \cdot \mu \text{m}^3.
  \end{split}
\end{equation}

\subsection{Second Term}
The nearly degenerate molecular states which contribute the most to the blockade shift for the second term in Equation \ref{eq_vdd_reinhard} are:
%\begin{equation}
%  \label{eq_rydmol3tru6}
%  \begin{align}
%	97D_{5/2} + 97D_{5/2} &\leftrightarrow 99P_{3/2} + 98P_{3/2},
%	97D_{5/2} + 97D_{5/2} &\leftrightarrow 98P_{3/2} + 98P_{3/2},
%	97D_{5/2} + 97D_{5/2} &\leftrightarrow 96F_{7/2} + 96F_{7/2},
%	97D_{5/2} + 97D_{5/2} &\leftrightarrow 95F_{7/2} + 96F_{7/2}
%  \end{align}
%\end{equation}
Using Mark's Rydebrg energy level calculations \cite{}, the F\"{o}rster defects are $\delta_{(1,2,3,4)}=(3.01,-4.46,4.94,-2.62)$ GHz respectively.


\section{$97S_{1/2}$}
